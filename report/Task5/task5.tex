\section{3D Geometry}

\paragraph{Task 5.1} The first step in this task is to find the epipolar lines of two images and then transform them via stereo rectification. The two images chosen did not contain epipoles due to how the camera moved between photos. The result of the rectification is shown in \autoref{fig:5_rectified}. The rectification process parallelises the epipolar lines and makes them horizontal in both images.

\begin{figure}
   \centering
   \begin{subfigure}{0.49\linewidth}
       \centering
       \includegraphics[width=0.9\linewidth]{Task5/images/rectified_1.png}
       \caption{First}
       \label{subfig:5_rectified:1}
   \end{subfigure}
   \begin{subfigure}{0.49\linewidth}
       \centering
       \includegraphics[width=0.9\linewidth]{Task5/images/rectified_2.png}
       \caption{Second}
       \label{subfig:5_rectified:2}
   \end{subfigure}
   \caption{Stereo rectified images}
   \label{fig:5_rectified}
\end{figure}

\paragraph{Task 5.2} Taking the disparity between the two rectified images gives an impression of depth, as a larger disparity suggests the those features are in the foreground, while a small disparity implies the feature is further away. This is achieved by computing the Census transform of the rectified stereo pair, before computing the hamming distance between pixels, followed by applying the semi-global matching method \cite{disparity_matlab}. The depth map generated from this process is shown in \autoref{fig:5_depthMap}.

\begin{figure}
   \centering
   \includegraphics[width=\linewidth]{Task5/images/depthMap.png}
   \caption{Depth map - larger disparity (red) indicates closer to the camera}
   \label{fig:5_depthMap}
\end{figure}